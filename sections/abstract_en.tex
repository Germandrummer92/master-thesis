%% LaTeX2e class for student theses
%% sections/abstract_en.tex
%% 
%% Karlsruhe Institute of Technology
%% Institute for Program Structures and Data Organization
%% Chair for Software Design and Quality (SDQ)
%%
%% Dr.-Ing. Erik Burger
%% burger@kit.edu
%%
%% Version 1.1, 2014-11-21

\Abstract
Today, speech recogntion has become ubiquitous in human machine interfaces. Be it voice control, speech translation or personal assistants, the speech recognition component always relies on the source language being known beforehand. This is due to the fact, that single-language speech recognizers perform better than multilingually trained ones. Language identification could therefore improve usability and performance of speech-based technology, without requiring any human interaction of selecting the correct source language. 

This thesis investigates a neural-network based approach to language identification. As all technologies mentioned above are used online, with a continous input signal, the focus here also lies on approaches feasible in an online environment. Current approaches to language identification are presented and compared. The results in this thesis compare different network structures, different audio preprocessing and network-postprocessing. Three different data corpora are used to verify results with different data sets. The best neural network structure gives a relative improvement of 18\% over the neural network employed in previous work.

Afterwards we eva

