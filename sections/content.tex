%% LaTeX2e class for student theses
%% sections/content.tex
%% 
%% Karlsruhe Institute of Technology
%% Institute for Program Structures and Data Organization
%% Chair for Software Design and Quality (SDQ)
%%
%% Dr.-Ing. Erik Burger
%% burger@kit.edu
%%
%% Version 1.1, 2014-11-21

%To be able to reference labels in other file
\externaldocument{introduction}
\externaldocument{appendix}

\chapter{Experimental Setup}
\label{ch:LITasks}

This chapter lays out the experimental setup used in this thesis. While Language Identification is applicable in many different scenarios, here the focus lies on trying to establish a low-latency online approach for recognizing the spoken language in a university-lecture environment. Because finding a suitable test setup for online data retrieval is hard, the data used was cut to short lengths to make an evaluation as to correctness of the recognition possible in an ''online-like'' scenario. 

This means that the output of the net is evaluated after short samples of speech and therefore can be seen as indicative of online performance of the net in the lecture translator

\section{Euronews 2014}
\label{sec:LITasks:Euronews}


The first data set used, was retrieved from Euronews \footnote{Euronews: http://www.euronews.com/} 2014. Euronews is a TV channel that is broadcast in 13 different languages simultaneously both on TV and over the Web and is semi-automatically transcribed. The data corpus includes 10 languages (Arabian, German, Spanish, French, Italian, Polish, Portuguese, Russian, Turkish and English) with around 18 hours of data per language provided overall. Data was taken both from online video and recordings of the transmissions as described in \cite{gretter2014euronews}.

It was then broken down to a per-speaker-basis based on its automatic transcriptions. From this data we took a random sample of 10.000 speakers, while making sure the total length of samples for each language were roughly the same. Details of this breakdown can be seen in table ~\ref{tab:spkData}.


\begin{table}[h!]
\label{tab:spkData}
\centering
\begin{tabular}{| l | c | r | }
	\hline
	\textbf{Language} & \textbf{Number of Speakers} & \textbf{Combined Length} \\
	\hline
	Arabian & 1055 & 16.76 h \\
	German & 928 & 18.80 h \\
	Spanish & 932 & 18.78 h \\
	French & 1016 & 18.67 h \\  
	Italian & 935 & 19.00 h \\  
	Polish & 1229 & 18.30 h \\ 
	Portuguese & 1062 & 16.19 h \\ 
	Russian & 958 & 18.66 h \\ 
	Turkish & 957 & 18.61 h \\  
	English & 928 & 18.54 h \\ 
	\hline
	\textbf{Overall} & 10000 & 182.31 h\\
	\hline
	
\end{tabular}
\caption{The Euronews corpus speaker breakdown with total utterances length}
\end{table}

\begin{table}[h!]
\label{tab:spkSplit}
\centering
\begin{tabular}{| l | c | r | }
	\hline
	\textbf{Set} & \textbf{Number of Speakers} & \textbf{Combined Length} \\
	\hline
	Train &  8000 & 149.76 h \\
	Development & 1000 & 19.46 h \\
	Test & 1000 & 19.29 h \\
	\hline
\end{tabular}
\caption{The Euronews corpus breakdown into the three data sets.}
\end{table}

The speaker list was then split into three smaller datasets: the train set, development set and test set using the common Simple Random Sampling. The sizes were \(80\%\) allocated to the train set, and \(10\%\) for both the development and test set. Table ~\ref{tab:spkSplit} shows the split data for the three sets. 

We also had a second Euronews Corpus available that was much larger. This was used to confirm intuition that a larger data corpus leads to better results which can be seen in Sec.~\ref{sec:LIDNetwork:Results}. The breakdown of the large corpus' data is listed in Table~\ref{tab:spkDataBig}

\newpage
\section{Lecture Data}
\label{sec:LITasks:Lecture}

As part of the development of the KIT's Lecture Translator, German lectures at the KIT were recorded and annotated. This is described in~\cite{stuker2012kit}. This thesis then uses parts of this German corpus as well as newer recordings done at the KIT of English lectures with the same setup, English academic talks given at InterACT - 25\footnote{InterACT 25: \url{http://www.interact25.org/}},  as well as French talks done at the DGA's~\footnote{DGA: \url{http://www.defense.gouv.fr/dga}} yearly academic conference on speech recognition. 

This lecture data was then used in two different ways: Firstly, to evaluate the 10-Language trained Euronews-Net(s) to see how it would fare in a lecture-environment as part of the KIT's lecture translator by scaling the output down to the 3 available languages. Secondly to train a second net and further try out the findings about the net setup and net evaluation, as found with the Euronews corpus.

The breakdown of speakers and length can be seen in Table~\ref{tab:spkDataLD}. As the main work was done on the Euronews corpus this data corpus is considerarbly smaller and was mostly just used as a proof-of-concept for a possible integration of a LID-Net into the Lecture Translator. 
\begin{table}[h!]
\label{tab:spkDataLD}
\centering
\begin{tabular}{| l | c | r | }
	\hline
	\textbf{Language} & \textbf{Number of Speakers} & \textbf{Combined Length} \\
	\hline
	German & 8 &  16.22 h \\
	French & 30 & 8.25 h \\  
	English & 27 & 10.78 h \\ 
	\hline
	\textbf{Overall} & 65 & 35.25 h\\
	\hline
	
\end{tabular}
\caption{The Lecture Data corpus speaker breakdown with total utterances length.}
\end{table}

\section{European Parliament}
\label{sec:LITasks:EU}

As another layer of evaluation we used recordings of the Europen Parliament speeches that are freely available online~\footnote{EU-Parliament plenary speeches: \url{http://www.europarl.europa.eu/ep-live/en/plenary/}}. The video recordings come with the simultaneous translations into all the official languages of the EU-countries. This includes seven of the languages also available on Euronews, namely German, English, French, Spanish, Italian, Polish and Portuguese. We extracted the seven audio tracks embedded in the recordings with ffmpeg~\footnote{ffmpeg:\url{https://ffmpeg.org/}} and evaluated performance of our Lecture Data as well as Euronews trained nets on this data.

\section{Feature Preprocessing}
\label{ch:FP}

This chapter deals with the feature preprocessing used to form normal speech into feature vectors to be understood by neural networks. The setup we used is based on the standard capabilities of the Janus Recognition Toolkit. It is then run through a six layer Automatic Speech Recognition network that was pre-trained on 10 languages. The 2\textsuperscript{nd} last layer of the ASR net is a Bottleneck feature layer, where the feature vectors are extracted and then used as input for the trained Language Identification Network. The following sections describe this Feature Preprocessing for data as well as the first ASR network used to create the BNF features the LID net requires.

\subsection{Feature Access}
\label{sec:FP:FA}

\subsection{Feature Description}
\label{sec:FP:FD}
%% TODO: references
The extracted ADC features from the audio files are then used for further preprocessing. We first use a standard Mel filter bank to extract only the necessary coefficients from the ADC0 feature. 

First a spectrum is applied to the ADC0 Feature, therefore calculating the Fast Fourier Transformation of the Digitalized Signal.

\subsection{ASR BNF network}
\label{sec:FP:Net}


\chapter{LID Network Structure and Results}
\label{ch:LIDNetwork}

This chapter describes the actual Language Identification Neural Network trained as well as the results of different network/data setups used. Most network experiments in the Network Geometry, meaning the number of hidden layers as well as the layout of the neurons in these layers were tried using the Euronews corpus. Results from this corpus were then transferred over to the other corpusses, meaning that the network layout that worked best for Euronews was then adjusted for the lecture data but otherwise the geometry was kept intact.

\section{Basic Setup}
\label{sec:LIDNetwork:Basic}

Many different tools exist for deep learning, the most acclaimed being Tensorflow~\cite{DBLP:journals/corr/AbadiABBCCCDDDG16}, DL4J/ND4j\footnote{DL4J:~\url{https://deeplearning4j.org/index.html}} and Theano~\cite{bergstra2011theano}. Pre-existing work on Language Identification using ASR BNFs used a python wrapper around Theano for training, that was developed by Jonas Gehring~\cite{gehringMA}. This thesis continues the use of this wrapper. The basic layout of the network used and improved upon by this thesis were input vectors from the ASR net~\ref{sec:FP:FA} with 966 coefficients. The net setup were 5 layers of denoising auto-encoders with each 1000 nodes and a tanh activation function using the mean squared error as the loss function. 

The neural net was then trained using mini batches of size 2m and a learning Rate of 0.01. The pretrained net was then retrained with a 1000 neuron to 10 coefficient output to get to our 10 Languages as classes to classify against. In the basic setup this was trained using a learning rate of 1 and the exit condition of a minimum change of 0.005 / 0.0001 for the training/validation data respectively.

The beginning benchmark to improve upon was then set to the frame-based validation/train error of 0.23 / 0.27 respectively, while of course understanding that non-training per-sample data would most likely have worse results at first than the frame-based validation error.

The following sections describe different network layouts and changes we made to the training of the neural network and the improvements we managed to make upon our initial result.

\section{Improving Network Layout}
\label{sec:LIDNetwork:Layout}

Different experiments were undertaken with the network layout. This includes a 6 hidden layer pre-training as well as a change in the geometry. The differences in the frame-based errors can be seen in Table~\ref{tab:LIDLayout}. 

\section{Results}
\label{sec:LIDNetwork:Results}


\chapter{Smoothing and Evaluation}
\label{ch:eval}

While frame-based error rates on the training/validation sets were already sufficiently good from the LID networks, this of course is not a reliable indicator of real-world online performance, so the development setup consisted of (for each data corpus) a development set of speakers whose samples were run through the LID setup (Feature Preprocessing Sec.~\ref{ch:FP} $\rightarrow$ ASR BNF extraction Sec.~\ref{sec:FP:Net} $\rightarrow$ LID network Sec.~\ref{ch:LIDNetwork}). The following smoothing approaches were applied in an ''online'' fashion, meaning it was made sure they can be calculated ``on-the-fly'' while new data is still coming in.

\section{Basic Test Filter}
\label{sec:eval:basic}

The first filter tried was a basic 5-Frame smoothing: It saves the value of the last direct outputs and only outputs a language if the last 5 direct outputs would have been the same. It also includes a filtering based on the actual output of the language ID neuron, only counting outputs higher than that. This of course means that the approach requires a 5 frame (\(\approx 50 ms\)) ``warm-up'' time, which still would make it usable in an online environment. It did however provide no improvement over the bare network output.

See Lst.~\ref{lst:basic} for the corresponding tcl/tk code for this basic filtering approach. The test setup used then goes through the entire development set of samples and counts the correctly/wrongly classified samples. This means that an extra amount of smoothing is included, but results should still be sufficiently general to be able to infer properties of employed filters, as they all include this extra smoothing. Table~\ref{tab:eval} shows a comparison of the basic filter with a bare setup only outputting the direct output of the LID net. Fig.~\ref{fig:lengthBasic} shows a comparison of the length of samples and the amount of correctly classified samples of this length for the LID net with the best evaluation results, the 6-layered geometry-adjusted lower learning rate net (See Sec.~\ref{ch:LIDNetwork:Layout}) 
\definecolor{green}{rgb}{0.0, 0.4, 0.13}
\lstset{captionpos=b,tabsize=3,frame=lines,keywordstyle=\color{blue},commentstyle=\color{green},stringstyle=\color{red},numbers=left,numberstyle=\tiny,numbersep=5pt,breaklines=true,showstringspaces=false,basicstyle=\footnotesize,emph={label}}
\begin{lstlisting}[label=lst:basic,caption=Most basic filter employed to smooth output]
proc filter{} {
    #setting up variables
     for {set i 0} {$i < 10} {incr i} {
        set totalM($i) 0
    }
    set lastFrameID -1
    set counter 0
    set currentOutput -1
    #Going through whole sample frame by frame in output layer of nn called nnBNF-> can be changed to work on continuously incoming data easily
    for {set i 0} { $i < [featureSetLID frameN nnBNF]} {incr i} {
        #we find the current output of the net
        set maxFrame [lindex [lsort -decreasing -real [featureSetLID frame nnBNF $i]] 0]
        set maxFrameID [lsearch -real [featureSetLID frame nnBNF $i] $maxFrame]
        #Actual Filtering: Only count if last 4 frames were also classified to be this language
        if {$maxFrameID != $lastFrameID} {
           set lastFrameID $maxFrameID
           set counter 0
        #Also filter for the actual output of the neuron
        } elseif {$maxFrame >= 0.61} {
           incr counter
           if {$counter >= 5} {
              set currentOutput $maxFrameID
           }
        }
        #setting total classification amounts for current sample
        if {$currentOutput != -1} {
             set totalM($currentOutput) [expr {[set totalM($currentOutput)] + 1}]
        }
    }
    set maxOverall -1
    set maxID -1
    #Now get the output for the whole sample (smoothing
    for {set i 0} {$i < 10} {incr i} {
        if {[set totalM($i)] > $maxOverall} {
            set maxOverall [set totalM($i)]
            set maxID $i
        }
    }
    #print out the total classification for the entire sample
    puts -nonewline "Overall we have classified as: "
    #help function to print language name not id.
    puts [getName $maxID]
    return $maxID
}

\end{lstlisting}


\section{Advanced Test Filter}
\label{sec:eval:advanced}
The advanced test filter is based on the jrtk's \textit{FILTER} capability. It automatically takes a defined amount of frames and calculates the weighted arithmetic mean with predefined weights for incoming audio. First tries were done using a basic filter setup of:
\begin{verbatim}
filter          nnFILTER   nnBNF  {-2 {1 2 3 2 1}}
\end{verbatim}  Herein the context is 2 frames on each side of the current frame (the first parameter) with weights 1, 2 , 3 , 2 , 1 for the 5 frames respectively.  It improved the evaluation results, but only by a negligible amount.

It was adjusted to use a 10-frame based approach with descending weights for frames further out:
\begin{verbatim}
filter          nnFILTERREAL nnBNF {-5 {1 2 3 4 5 6 5 4 3 2 1}}
\end{verbatim}
which however did not provide a furhter increase in correctness. The full tcl code for this can be found in the Appendix.


\section{Variance Test Filter}
\label{sec:eval:variance}
As a next approach, the variance between the two most likely outputs was taken into account. This however, did not lead to an improvement in the robustness of the classification on the development set. The reason possibly being that the output of the 2\textsuperscript{nd} most likely neuron is not going to differ from the maximum output on many different language pairs: E.g. French/Italian, Russian/Polish, Italian/Portuguese. The full tcl code of this filter can be found in the appendix . The evaluation results can be seen in Tab.~\ref{tab:eval}

\section{Two-Language setup}
\label{sec:eval:2L}
Table~\ref{tab:eval:2l} shows the result produced by using the LID Euronews net for 10 languages on different combinations of 2 languages (french/italian and english/german). In this case we ignore the output of the net if it doesn't equal one of the two languages and instead keep the previous output intact in this case. This, as intuition predicted, gave a big boost in the recognition rate, bringing the rate up to 85 \% for the two languages combined, an improvement of more than 10 \% in correctness compared to the basic approach in Sec.~\ref{sec:eval:basic}.

	