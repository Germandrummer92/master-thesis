%% LaTeX2e class for student theses
%% sections/abstract_de.tex
%% 
%% Karlsruhe Institute of Technology
%% Institute for Program Structures and Data Organization
%% Chair for Software Design and Quality (SDQ)
%%
%% Dr.-Ing. Erik Burger
%% burger@kit.edu
%%
%% Version 1.1, 2014-11-21

\Abstract

Heutzutage ist Automatische Spracherkennung (ASR) aus der Mensch-Maschinen-\newline Kommunikation (MMK) nicht mehr wegzudenken. Ob es Sprachkontrolle, Sprachübersetzung oder persönliche Assistenten auf dem Smartphone sind, der ASR Teil beruht darauf,  dass die Eingangssprache vor der Erkennung bekannt ist. Dies verbessert die Leistung des Spracherkenners, da auf einer Sprache trainierte Spracherkenner eine geringere Fehlerrate haben als multilingual-trainierte. Daher kann die Sprachenidentifizierung (LID) in dieser MMK die Benutzerfreundlichkeit und Leistung verbessern. Somit wird keine Aktion des Benutzers mehr erforderlich. Diese MMK ist in der heutigen Technologie immer auf Online-Performanz und Geschwindigkeit bedacht, daher evaluieren wir alle unsere Ansätze unter diesem Gesichtspunkt.

Diese Arbeit untersucht einen Ansatz der LID, der Neuronale Netze benutzt. Auch stellen wir heutige Ansätze der LID der Literatur dar und vergleichen sie mit dieser Arbeit. Wir haben verschieden Netzwerkstrukturen, verschiedene Audio-Vorverarbeitungen und Netzwerk-Nachverarbeitungen evaluiert. Drei verschiedene Datensätze werden eingeführt auf denen wir alle unsere Ergebnisse testen. Die Netzstruktur mit dem besten Ergebniss resultierte in einer relativen Verbesserung von 18 \% gegenüber unserem ursprünglichen Aufbau.

Wir haben auch verschiedene Nachverarbeitungen der Neuronalen Netz-Ausgabe verglichen. Hierfür führen wir unser eigene Metrik den ``Out-of-Language-Error'' (OLE) ein, der das ``Rauschen'' der einzelnen Filter/des Netzes misst. Insgesamt produzieren ein einfacher Zählfilter, ein maximal-Sequenz Filter und ein Gauß-Filter die besten Ergebnisse und könnten die Ergebnisse eines LID Netzes in einer konkreten Implementierung stark verbessern.